\documentclass{article}

% Package Pygments for fancy typesetting of code
\usepackage{fancyvrb}

\usepackage[T1]{fontenc}
\usepackage[utf8]{inputenc}
\usepackage{color}

% Hyperlinks in PDF: \href{url}{linktext}
\definecolor{linkcolor}{rgb}{0,0,0.4}
\usepackage[%
    colorlinks=true,
    linkcolor=linkcolor,
    urlcolor=linkcolor,
    citecolor=black,
    filecolor=black,
]{hyperref}

\begin{document}

\title{\LaTeX{} rapport til  oblig 1}

\author{
Arselan Sultani	\footnote{\texttt{arselans@ulrik.uio.no}.}
}

\date{\today}

% Generate title, author, and date
\maketitle

\begin{abstract}
Denne rapporten skal leveres angående obligatorisk innlevering 1 i
INF3331 og den er skrevet i \LaTeX{} . Denne skal leveres sammen med 
shell scriptene og python scriptet på Github. 
\end{abstract}

\section{Del 1}
\label{sec:Del 1}

\subsection{list\_new\_files.sh}
\label{sec:1.1}
Den finner de filene som er endret i n siste dager og mappen som 
dn skal lete i. Navnet på mappen og n får vi fra brukeren som argumenter.
-mtime 2.arg: sier om antall dager 
-print0:  er dersom der er mellomrom i navn på filen, fordi ellers
vil den leste samme fil som to forskjellige navner.
-B k: er for a få med K bak fil størrelsen

\subsection{find\_word.sh}
\label{sec:1.2}
Her skal vi lete i filer i mappen gitt, etter en ord som blir 
gitt av brukeren.
-grep 2.argument: er ordet vi skal finne i de filene


\subsection{sized\_delete.sh}
\label{sec:1.3}
Her skal vi finne alle de filene som er større enn en gitt 
størrelse og slette de.
-size 2.arg: vil si størrelsen av filen vi søker etter er større enn 
2.arg
-exec rm: vil si at den skal slette uten å spørre om tillatelse fordi
dersom vi hadde sagt -ok rm, ville den ha funnet de filene men spurt 
en og en om vi ville fjerne de. 


\subsection{sort\_file.sh}
\label{sec:1.4}
Her skal vi lese ord fra en fil og sortere de og så skrive ut til
en annen fil. Begge filene blir gitt av brukeren. Og dersom output
filen ikke eksisterer, vil den bli opprettet og skrevet i.
-o 2.arg er output-filen

\section{Oppgave 2}
\label{sec:Oppg.2}

Den starter ved a sjekke argumenter, og hvis det er mindre enn 4, så 
skriver den ut en feilmelding. Hvis ikke, så skriver den ut først 
target, så files osv. Så kaller vi på metoden generate\_tree.

\subsection{generate\_tree() og mkDirs()}
Jeg har antatt at vi skulle finne random dypde fra root. Og det skal
være mellom 2 og det er som gitt. Så skal den finne random tall mellom 2 
til dir fordi jeg har antall at også dette skal være random. Så kaller den
på mkDirs metoden som lager mappene.
Så sier at så lenge dypde er større eller lik 0, så skal den gå videre.
Så lenge x er mellom 1 og en random tall for antall mapper i mappen,
lager jeg en ny path i new\_dir2. Den tar den pathen som den allerede har og
legger en \textbackslash i mellom og så en tilfeldig string fra random\_string() som 
blir navnet på den nye mappen. os.makedirs lager mappen. Deretter blir 
det sjekket om verbose er True eller False, for dersom den er True, så skal 
den skrive pathen til den nye mappen. Så kalles populate\_tree(), som fyller 
mappene med filer. Og den vil gå i rekursjon. Den vil eventuelt bli stoppet
av for-løkka og dir\_deep som minsker hver gang vi lager en mappe inn i en
mappe fordi da er vi lenger unna root.

\subsection{populate\_tree()}
Den vil først lage en last\_modified og last\_accessed. last\_modified skal
være mellom start\_time og end\_time, som vi får fra argumentene og 
last\_accessed skal være mellom last\_modified og end\_time, fordi jeg
antar at du kan få tilgang til filen uten å ha endret på den. 
Deretter lages det filer i mappen. Filnavnet blir opprettet og vi spør 
om vi kan åpne filen. Og dersom filen eksisterer, så kan skal vi åpne
den og overskrive. Men dersom filen ikke eksisterer, så oppretter vi den. 
Dersom verbose er True, så skriver vi ut pathen til filen som vi nettopp
lagde. Deretter finner vi størrelsen til filen og fyller den etter til
det blir så mye som den vil ha Og til slutt closer vi den.


\end{document}


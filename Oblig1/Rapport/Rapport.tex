\documentclass{article}
\usepackage{fancyvrb}
\usepackage[T1]{fontenc}
\usepackage[utf8]{inputenc}
\title{Obligatorisk innlevering 1 høsten 2014, INF3331}
\author{Arselan Sultani <arselans@ulrik.uio.no>}
\date{\today}
\begin{document}
\maketitle
\catcode`\_=\active
\def_{\_}%

\section{Oppgave 1.1}
Oppgaven løst ved bruk av følgende komandoer: 
\\* find: for å finne de filene vi leter etter
\\* -type: type fil vi leter etter
\\* -mtime: alder på filen vi leter etter
\\* xargs: for å vise filresultatene
\\* sort: Sort resultatet

\subsection{Kjøring}
\begin{Verbatim}
./list_new_files.sh file_tree/ 100
34      file_tree/Pkvye/htZiVgRE
45      file_tree/Kq0Wv/5RYWI5kQ
184     file_tree/zg/grYxji7
564     file_tree/zg/Hu/dNmOlK
644     file_tree/Kq0Wv/MH/XhdhBbk

\end{Verbatim}
Alle file i mappen blir skrevet ut dersom ingen av filene passer 
kriteriene som er gitt av brukeren. 


\section{Oppgave 1.2}
Oppgaven løst ved bruk av følgende komandoer: 
\\* find: for å finne de filene vi leter etter
\\* grep: er ordet som vi får som vi skal lete etter

\subsection*{Kjøring}
\begin{Verbatim}
./find_word.sh file_tree/ what
file_tree/Pkvye/vlfN/ZLbGhCmj:NDa6gmZswhat77iTUFuoNiG23Y

./find_word.sh file_tree/ hello

\end{Verbatim}

Når den ikke finner noe, blir ingenting skrevet ut.

\section*{Oppgave 1.3}
Oppgaven løst ved bruk av følgende komandoer: 
\\* find: for å finne de filene vi leter etter
\\* -size: størrelsen som er større enn gitt
\\* -exec rm -rf {}  \textbackslash ; Fjerner alle filer som 
oppfyller kriteriene. Dersom vi hadde hatt -ok i stedet for -exec
hadde den spurt oss for hver fil om vi vil slette eller ikke. Mens 
-exec bare utfører.  

\subsection*{Kjøring}
\begin{Verbatim}
./sized_delete.sh file_tree/ 750
file_tree/Kq0Wv/MH/Z9kP8NB
file_tree/Kq0Wv/MH/zWG/8puxfjS
file_tree/Pkvye/vlfN/ZLbGhCmj
file_tree/zg/Hu/vv/2KKnyIt5

\end{Verbatim}


\section*{Oppgave 1.4}
Oppgaven løst ved bruk av følgende komandoer: 
\\* sort: Sortere filen som man får i første argumentet
\\* -o: Skrive til fil som man får som andre argument


\subsection*{Kjøring}
\begin{Verbatim}
./sort_file.sh unsorted_fruits sorted_fruits

cat sorted_fruits
apple
grape
orange
pear
pineapple

\end{Verbatim}


\section{Oppgave 2}

Den starter ved a sjekke argumenter, og hvis det er mindre enn 4, så 
skriver den ut en feilmelding. Hvis ikke, så skriver den ut først 
target, så files osv. Så kaller vi på metoden generate\_tree.

\subsection{generate\_tree() og mkDirs()}
Jeg har antatt at vi skulle finne random dypde fra root. Og det skal
være mellom 2 og det er som gitt. Så skal den finne random tall mellom 2 
til dir fordi jeg har antall at også dette skal være random. Så kaller den
på mkDirs metoden som lager mappene.
Så sier at så lenge dypde er større eller lik 0, så skal den gå videre.
Så lenge x er mellom 1 og en random tall for antall mapper i mappen,
lager jeg en ny path i new\_dir2. Den tar den pathen som den allerede har og
legger en \textbackslash i mellom og så en tilfeldig string fra random\_string() som 
blir navnet på den nye mappen. os.makedirs lager mappen. Deretter blir 
det sjekket om verbose er True eller False, for dersom den er True, så skal 
den skrive pathen til den nye mappen. Så kalles populate\_tree(), som fyller 
mappene med filer. Og den vil gå i rekursjon. Den vil eventuelt bli stoppet
av for-løkka og dir\_deep som minsker hver gang vi lager en mappe inn i en
mappe fordi da er vi lenger unna root.

\subsection{populate\_tree()}
Den vil først lage en last\_modified og last\_accessed. last\_modified skal
være mellom start\_time og end\_time, som vi får fra argumentene og 
last\_accessed skal være mellom last\_modified og end\_time, fordi jeg
antar at du kan få tilgang til filen uten å ha endret på den. 
Deretter lages det filer i mappen. Filnavnet blir opprettet og vi spør 
om vi kan åpne filen. Og dersom filen eksisterer, så kan skal vi åpne
den og overskrive. Men dersom filen ikke eksisterer, så oppretter vi den. 
Dersom verbose er True, så skriver vi ut pathen til filen som vi nettopp
lagde. Deretter finner vi størrelsen til filen og fyller den etter til
det blir så mye som den vil ha Og til slutt closer vi den.




\end{document}